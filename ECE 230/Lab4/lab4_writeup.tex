\documentclass[12pt]{article}
%-------------PACKAGES-------------%
\usepackage[letterpaper, margin=1.0in]{geometry} % page layout
\usepackage[utf8]{inputenc} % input encoding
\usepackage[T1]{fontenc} % font encoding
\usepackage{parskip} % paragraph formatting
\usepackage{fancyhdr} % header formatting
\usepackage{amsmath} % math features
\usepackage{mathtools} % math formating
\usepackage{amssymb} % math symbols
\usepackage{siunitx} % SI units
\usepackage{graphicx} % images
\usepackage{caption} % captions
\usepackage{multirow} % combine rows in tables
\usepackage{xcolor} % colors
\usepackage[american,straightlabels,nooldvoltagedirection]{circuitikz} % drawing circuit diagrams
\usepackage{subcaption}
%-------------FORMAT-------------%
\pagestyle{fancy}
\renewcommand{\headrulewidth}{0.5mm}
\renewcommand{\footrulewidth}{0.5mm}
\setlength{\headheight}{14.5pt}
\newcommand{\makeheader}[2]{ % Takes argument {Lab #}{Title}
\begin{center}
\textbf{\huge ECE 230L - \MakeUppercase{#1}}\\~\\
\textbf{\large \MakeUppercase{#2}}\\~\\
\rule{6.5in}{0.5mm}\\
\end{center}
\fancyhead[L]{ECE230L}
\fancyhead[C]{}
\fancyhead[R]{Duke University}
\fancyfoot[L]{Lab Report}
\fancyfoot[C]{#1}
\fancyfoot[R]{Page \thepage}
}
\begin{document}
%------Create header/footer------%
\makeheader{Lab 4}{PSPICE CIRCUIT SIMULATION}
\setcounter{section}{3}
\setcounter{subsection}{0}
\setcounter{subsubsection}{0}
\section{DC Analysis in PSpice}
$\square$ Take a screenshot of your circuit and plot.
\begin{figure}[ht!]
\centering
\includegraphics[width=0.9\textwidth]{dc_circuit.png}
\caption{Circuit for DC Analysis}
\end{figure}
\begin{figure}[ht!]
\centering
\includegraphics[width=0.9\textwidth]{dc_waveform.png}
\caption{Result of DC Analysis.}
\end{figure}
\clearpage
\section{AC Analysis in PSpice}
$\square$ Take a screenshot of your circuit and plot.
\begin{figure}[ht!]
\centering
\includegraphics[width=0.9\textwidth]{ac_circuit.png}
\caption{Circuit for AC Analysis}
\end{figure}
\begin{figure}[ht!]
\centering
\includegraphics[width=0.9\textwidth]{ac_waveform.png}
\caption{Result of AC Analysis}
\end{figure}
\clearpage
\section{Transient Analysis in PSpice}
$\square$ Take a screenshot of your circuit and plot.
\begin{figure}[ht!]
\centering
\includegraphics[width=0.9\textwidth]{transient_circuit.png}
\caption{Circuit for Transient Analysis}
\end{figure}
\begin{figure}[ht!]
\centering
\begin{center}
\includegraphics[width=0.9\textwidth]{transient_waveform.png}
\end{center}
\caption{Result of Transient Analysis}
\end{figure}
\clearpage
\section{Practice Example}
$\square$ Include a screenshot of the circuit for one of the analyses
\begin{figure}[ht!]
\centering
\includegraphics[width=0.9\textwidth]{example_circuit.png}
\caption{Circuit for DC Analysis}
\end{figure}

$\square$ Include screenshots of the plots of voltage across the capacitor $C_1$ for all 3 of the analyses

\begin{figure}[ht!]
\centering
\includegraphics[width=0.9\textwidth]{example_dc_waveform.png}
\caption{Result of DC Analysis.}
\end{figure}
\begin{figure}[ht!]
\centering
\includegraphics[width=0.9\textwidth]{example_ac_waveform.png}
\caption{Result of AC Analysis}
\end{figure}
\begin{figure}[ht!]
\centering
\includegraphics[width=0.9\textwidth]{example_transient_waveform.png}
\caption{Result of Transient Analysis.}
\end{figure}
\clearpage
\section{Exploration: Th\'evenin Equivalent Circuits}
\subsection{Example Exercise}
In your report, include images of each PSpice circuit showing all voltages and
currents
\begin{itemize}
\setlength\itemsep{-0.5em}
\item[$\square$] The original circuit
\item[$\square$] The open circuit
\item[$\square$] The short circuit
\item[$\square$] The Th\'evenin Equivalent circuit
\end{itemize}
\begin{figure}[ht!]
\begin{subfigure}{0.5\textwidth}
\begin{center}
\includegraphics[width=0.9\textwidth]{thev_a.png}
\caption{Example Circuit}
\end{center}
\end{subfigure}
\hfill
\begin{subfigure}{0.5\textwidth}
\begin{center}
\includegraphics[width=0.9\textwidth]{thev_b.png}
\caption{Open Circuit Voltage Schematic}
\end{center}
\end{subfigure}
\begin{subfigure}{0.5\textwidth}
\begin{center}
\includegraphics[width=0.9\textwidth]{thev_c.png}
\vspace{0.7cm}
\caption{Short Circuit Voltage Schematic}
\end{center}
\end{subfigure}
\hfill
\begin{subfigure}{0.5\textwidth}
\begin{center}
\includegraphics[width=0.9\textwidth]{thev_d.png}
\caption{Th\'evenin Equivalent Schematic}
\end{center}
\end{subfigure}
\end{figure}
$\square$ Show how you extracted $V_{OC}$ and $I_{SC}$ for the circuit

Thevenin circuits work by replacing the original circuit with a voltage source $V_{TH}$ in series with a resistor $R_{TH}$. To find $V_{OC}$, we remove the load resistor and measure the voltage across the terminals. To find $I_{SC}$, we short the terminals and measure the current. In this case, $V_{OC} = 166.7mV$ and $I_{SC} = 588.2 \mu A$.

$\square$ Show how you determined $R_{TH}$

To find $R_{TH}$, we first remove all sources from the original circuit. We then find the resistance across the terminals. In this case, $R_{TH} = V_{OC}/I_{SC} = 166.7mV/588.2 \mu A = 283.3 \Omega$.

\newpage
\subsection{Challenge Exercise: Th\'evenin Equivalent Circuit}
In your report, include images of each PSpice circuit showing all voltages and
currents
\begin{itemize}
\setlength\itemsep{-0.5em}
\item[$\square$] The original circuit
\item[$\square$] The open circuit
\item[$\square$] The short circuit
\item[$\square$] The Th\'evenin Equivalent Circuit
\end{itemize}
\begin{figure}[ht!]
\begin{subfigure}{0.5\textwidth}
\begin{center}
\includegraphics[width=0.9\textwidth]{challenge_orig.png}
\caption{Full Circuit}
\end{center}
\end{subfigure}
\hfill
\begin{subfigure}{0.5\textwidth}
\begin{center}
\includegraphics[width=0.9\textwidth]{challenge_open.png}
\caption{Open Circuit Voltage Schematic}
\end{center}
\end{subfigure}
\begin{subfigure}{0.5\textwidth}
\begin{center}
\includegraphics[width=0.9\textwidth]{challenge_closed.png}
\vspace{0.7cm}
\caption{Short Circuit Voltage Schematic}
\end{center}
\end{subfigure}
\hfill
\begin{subfigure}{0.5\textwidth}
\begin{center}
\includegraphics[width=0.9\textwidth]{challenge_thev.png}
\caption{Th\'evenin Equivalent Schematic}
\end{center}
\end{subfigure}
\end{figure}
$\square$ Show how you extracted $V_{OC}$ and $I_{SC}$ for the circuit

Again, like before we remove the load resistor to find $V_{OC}$ and short the terminals to find $I_{SC}$. In this case, $V_{OC} = 909.1mV - 714.3mV = 194.8mV$ and $I_{SC} = 1.667mA - 833.3 \mu A = 833.3 \mu A$.


$\square$ Show how you determined $R_{TH}$

Again, we remove all sources from the original circuit and find the resistance across the terminals. In this case, $R_{TH} = V_{OC}/I_{SC} = 194.8mV/833.3 \mu A = 233.8 \Omega$.

\end{document}
