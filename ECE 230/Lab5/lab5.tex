\documentclass[12pt]{article}

%-------------PACKAGES-------------%
\usepackage[letterpaper, margin=1.0in]{geometry} % page layout
\usepackage[utf8]{inputenc} % input encoding
\usepackage[T1]{fontenc}    % font encoding
\usepackage{parskip}        % paragraph formatting
\usepackage{fancyhdr}       % header formatting
\usepackage{amsmath}        % math features
\usepackage{mathtools}      % math formating
\usepackage{amssymb}        % math symbols
\usepackage{siunitx}        % SI units
\usepackage{graphicx}       % images
\usepackage{caption}        % captions
\usepackage{multirow}       % combine rows in tables
\usepackage{xcolor}         % colors
\usepackage[american,straightlabels,nooldvoltagedirection]{circuitikz} % drawing circuit diagrams

%-------------FORMAT-------------%
\pagestyle{fancy}
\renewcommand{\headrulewidth}{0.5mm}
\renewcommand{\footrulewidth}{0.5mm}
\setlength{\headheight}{14.5pt}

\newcommand{\makeheader}[2]{ % Takes argument {Lab #}{Title}
    \begin{center}
    \textbf{\huge ECE 230L - \MakeUppercase{#1}}\\~\\
    \textbf{\large \MakeUppercase{#2}}\\~\\
    \rule{6.5in}{0.5mm}\\
    \end{center}
    
    \fancyhead[L]{ECE230L}
    \fancyhead[C]{}
    \fancyhead[R]{Duke University}
    \fancyfoot[L]{Lab Report}
    \fancyfoot[C]{#1}
    \fancyfoot[R]{Page \thepage}
}

\begin{document}
%------Create header/footer------%
\makeheader{Lab 5}{METAL-OXIDE-SEMICONDUCTOR FIELD-EFFECT TRANSISTOR (MOSFET) CHARACTERIZATION}


\setcounter{section}{1}
\section{Electrical Characterization of the MOSFET}

\begin{itemize}
\item[$\square$] $I_D(V_{\mathrm{DD}}, V_{\mathrm{GG}})$ plot

\begin{figure}[h]
    \centering
    \includegraphics[width=0.7\textwidth]{ID_1.png}
    \caption{$I_D(V_{\mathrm{DD}}, V_{\mathrm{GG}})$}
    \label{fig:IDDataPlot}
\end{figure}

\item[$\square$] $V_{\mathrm{DS}}(V_{\mathrm{DD}}, V_{\mathrm{GG}})$ plot

\begin{figure}[h]
    \centering
    \includegraphics[width=0.7\textwidth]{V_DS_1.png}
    \caption{$V_{\mathrm{DS}}(V_{\mathrm{DD}}, V_{\mathrm{GG}})$}
    \label{fig:VDSDataPlot}
\end{figure}

\item[$\square$] $V_{\mathrm{GS}}(V_{\mathrm{DD}}, V_{\mathrm{GG}})$ plot

\begin{figure}[h]
    \centering
    \includegraphics[width=0.7\textwidth]{V_GS_1.png}
    \caption{$V_{\mathrm{GS}}(V_{\mathrm{DD}}, V_{\mathrm{GG}})$}
    \label{fig:VGSDataPlot}
\end{figure}

\item[$\square$] $I_D(V_{\mathrm{GS}}, V_{\mathrm{DS}} = \SI{3.0}{\volt})$ plot

\begin{figure}[h]
    \centering
    \includegraphics[width=0.7\textwidth]{ID_const_1.png}
    \caption{$I_D(V_{\mathrm{GS}}, V_{\mathrm{DS}} = \SI{3.0}{\volt})$}
    \label{fig:IDConstantVDSDataPlot}
\end{figure}

\end{itemize}
\clearpage


\section{Exploration --- Infrared Receiver}

\subsection{Infrared Receiver}
\setcounter{enumi}{3}
\begin{enumerate}
\item[$\square$] \textbf{Take a photo} (or video!) of your completed and working receiver circuit.  This will be turned in as part of your assignment for this week's lab.
\end{enumerate}

\begin{figure}[h]
    \centering
    \includegraphics[width=0.7\textwidth]{buzzer.png}
    \caption{$I_D(V_{\mathrm{GS}}, V_{\mathrm{DS}} = \SI{3.0}{\volt})$}
    \label{fig:Buzzer}
\end{figure}



\section{Questions}
\begin{enumerate}
\item Give the definitions of the conduction parameter $K_N$ and the transconductance $g_m^\mathrm{sat}$ of the NMOSFET. Comment on their dependence on other NMOSFET parameters and bias voltages.


\item Give the definition of the NMOSFET channel-length-modulation parameter $\lambda$. Describe how it is extracted from the drain-current characteristics $I_D(V_{\mathrm{DS}}, V_{\mathrm{GS}})$.


\item Plot the dependence of the drain current $I_{D}$ on the drain-to-source bias voltage $V_{\mathrm{DS}}$ for different values of the gate-to-source bias voltage $V_{\mathrm{GS}}$ on a linear-linear plot. Don't forget that what was measured in lab was $I_D(V_{\mathrm{DD}}, V_{\mathrm{GS}})$ and $V_{\mathrm{DS}}(V_{\mathrm{DD}}, V_{\mathrm{GS}})$. In order to plot $I_D(V_{\mathrm{DS}}, V_{\mathrm{GS}})$, you will need to do a substitution. (The only reason $V_{\mathrm{GS}}(V_{\mathrm{DD}}, V_{\mathrm{GG}})$ was measured was to show that $V_{\mathrm{GS}} = V_{\mathrm{GG}}$)


\item Plot the dependence of $\sqrt{I_D}$ on the gate-to-source bias voltage $ V_{\mathrm{GS}}$ in the 0 to \SI{6}{\volt} range for a fixed value of the drain-to-source bias voltage $V_{\mathrm{DS}}$ equal to \SI{3}{\volt}.


\item From the above plots, obtain the values of $K_N$, $V_T$, and $\lambda$. Include any additional plots or calculations used to obtain these parameters. Note that the parameter extraction guidelines give two methods for finding $K_N$. For any values where multiple different data sets can be used, remember to take several measurements and average the results. In a table, compare (\% error) your extracted values with the following expected values: $K_N=0.1233\frac{A}{V^2}$, $V_t=1.824V$, $\lambda=0.03V^{-1}$.


\item What are the major sources of error in the values of the extracted parameters? How can these errors be effectively reduced?
\end{enumerate}

\section{Extension}

\end{document}